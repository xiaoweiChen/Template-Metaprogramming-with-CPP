This chapter was dedicated to seeing how templates can be used to build general-purpose libraries. Although we couldn’t cover these topics in great detail, we have explored the design of containers, iterators, and algorithms from the C++ standard library. These are the pillars of the standard library. We spent most of the chapter understanding what it takes to write a container similar to the standard ones as well as an iterator class to provide access to its elements. For this purpose, we implemented a class that represents a circular buffer, a data structure of fixed size where elements are overwritten once the container is full. Lastly, we implemented a general-purpose algorithm that zips elements from two ranges. This works for any container including the circular buffer container.

Ranges, as discussed in this chapter, are an abstract concept. However, that changed with C++20, which introduced a more concrete concept of ranges with the new ranges library.
This is what we will discuss in the final chapter of this book.