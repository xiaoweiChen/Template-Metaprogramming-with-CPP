Type traits are an important metaprogramming technique that enables us to inspect properties of types or to perform transformations of types at compile-time. Type traits are themselves templates and you can see them as meta-types. Knowing information such as the nature of a type, its supported operations, and its various properties is key for performing conditional compilation of templated code. It is also very useful when writing a library of templates.

In this chapter, you will learn the following:

\begin{itemize}
\item
Understanding and defining type traits

\item
Understanding SFINAE and its purpose

\item
Enabling SFINAE with the enable\_if type trait

\item
Using constexpr if

\item
Exploring the standard type traits

\item
Seeing real-world examples of using type traits
\end{itemize}

By the end of the chapter, you will have a good understanding of what type traits are, how they are useful, and what type traits are available in the C++ standard library.

We will start the chapter by looking at what type traits are and how they help us.



























