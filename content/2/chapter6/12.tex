The C++20 standard introduced some new major features to the language and the standard library. One of these is concepts, which was the topic of this chapter. A concept is a named constraint that can be used to define requirements on template arguments for function templates, class templates, variable templates, and alias templates.

In this chapter, we have explored in detail how we can use constraints and concepts and how they work. We have learned about requires clauses (that determine whether a template participates in overload resolution) and requires expressions (that specify requirements for well-formedness of expressions). We have seen what various syntaxes are for specifying constraints. We also learned about abbreviated function templates that provide a simplified syntax for function templates. At the end of the chapter, we explored the fundamental concepts available in the standard library.

In the next chapter, we will shift our attention toward applying the knowledge accumulated so far to implement various template-based patterns and idioms.