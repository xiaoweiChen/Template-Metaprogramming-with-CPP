第3章中,我们讨论了可变参数模板,其是用一种看起来像递归的机制实现的。实际上,是重载函数和类模板的特化。可以创建递归模板,为了演示其是如何工作的,我们将研究如何实现阶乘函数的编译时版本。这通常会以递归的方式实现,可能的实现如下所示:

\begin{lstlisting}[style=styleCXX]
constexpr unsigned int factorial(unsigned int const n)
{
	return n > 1 ? n * factorial(n - 1) : 1;
}
\end{lstlisting}

这应该很容易理解:通过递归地调用函数和递减的参数,返回函数参数与返回值相乘的结果,若参数为0或1,则返回值1。参数的类型(以及返回值)是unsigned int,以避免使用负整数。

为了在编译时计算阶乘函数的值,需要定义一个类模板,其中包含一个持有函数值的数据成员:

\begin{lstlisting}[style=styleCXX]
template <unsigned int N>
struct factorial
{
	static constexpr unsigned int value =
		N * factorial<N - 1>::value;
};

template <>
struct factorial<0>
{
	static constexpr unsigned int value = 1;
};

int main()
{
	std::cout << factorial<4>::value << '\n';
}
\end{lstlisting}

第一个定义是主模板,有一个非类型模板参数,表示需要计算其阶乘的值。该类包含一个名为value的静态constexpr数据成员,初始化的结果是参数N与阶乘类模板的值相乘,参数自减后实例化。递归需要一个初始值,这是由(非类型模板参数的)第一个参数(0号)的显式特化提供的,成员值初始化为1。

main函数中遇到实例化factorial<4>::value时,编译器生成从factorial<4>到factorial<0>的所有递归实例化:

\begin{lstlisting}[style=styleCXX]
template<>
struct factorial<4>
{
	inline static constexpr const unsigned int value =
		4U * factorial<3>::value;
};

template<>
struct factorial<3>
{
	inline static constexpr const unsigned int value =
		3U * factorial<2>::value;
};

template<>
struct factorial<2>
{
	inline static constexpr const unsigned int value =
		2U * factorial<1>::value;
};

template<>
struct factorial<1>
{
	inline static constexpr const unsigned int value =
		1U * factorial<0>::value;
};

template<>
struct factorial<0>
{
	inline static constexpr const unsigned int value = 1;
};
\end{lstlisting}

这些实例化中,编译器能够计算出数据成员阶乘<N>::value的值。当启用优化时,甚至不会生成此代码,但生成的常量将直接用于生成的汇编代码中。

阶乘类模板的实现相对简单,类模板基本上只是静态数据成员值的包装器。实际上,可以通过变量模板来避免这种情况:

\begin{lstlisting}[style=styleCXX]
template <unsigned int N>
inline constexpr unsigned int factorial = N * factorial<N - 1>;

template <>
inline constexpr unsigned int factorial<0> = 1;

int main()
{
	std::cout << factorial<4> << '\n';
}
\end{lstlisting}

阶乘类模板和阶乘变量模板的实现之间有很多相似之处。对于变量模板,我们取出了数据成员值,并将其称为阶乘。另一方面,因为不需要访问factorial<4>::value中数据成员的值,这也会容易使用。

编译时计算阶乘还有第三种方法:使用函数模板。可能的实现如下所示:

\begin{lstlisting}[style=styleCXX]
template <unsigned int n>
constexpr unsigned int factorial()
{
	return n * factorial<n - 1>();
}

template<> constexpr unsigned int factorial<1>() {
												return 1; }
template<> constexpr unsigned int factorial<0>() {
												return 1; }

int main()
{
	std::cout << factorial<4>() << '\n';
}
\end{lstlisting}

可以看到主模板递归地调用factorial函数模板,并且对值1和0有两个完全的特化,都返回1。

这三种不同的方法中哪一种是最好的可能不太好说,但阶乘模板的递归实例化的复杂性保持不变,这取决于模板的性质。下面的代码段增加了复杂性:

\begin{lstlisting}[style=styleCXX]
template <typename T>
struct wrapper {};

template <int N>
struct manyfold_wrapper
{
	using value_type =
		wrapper<
			typename manyfold_wrapper<N - 1>::value_type>;
};

template <>
struct manyfold_wrapper<0>
{
	using value_type = unsigned int;
};

int main()
{
	std::cout <<
		typeid(manyfold_wrapper<0>::value_type).name() << '\n';
	std::cout <<
		typeid(manyfold_wrapper<1>::value_type).name() << '\n';
	std::cout <<
		typeid(manyfold_wrapper<2>::value_type).name() << '\n';
	std::cout <<
		typeid(manyfold_wrapper<3>::value_type).name() << '\n';
}
\end{lstlisting}

本例中有两个类模板。第一个称为wrapper,有一个空实现(并不重要),但它表示某种类型的包装器类(或者更准确地说,某种类型的值)。第二个模板称为manyfold\_wrapper,表示一个包装器对一个类型的包装器进行多次包装。包装数量没有上限,但是下限有。0值的全特化为unsigned int类型定义了一个名为value\_type的成员类型。因此,manyfold\_wrapper<1>为wrapper<unsigned int>定义了一个名为value\_type的成员类型,manyfold\_wrapper<2>为wrapper<wrapper<unsigned int>>定义了一个名为value\_type的成员类型,以此类推。在执行main函数时,会将输出以下内容到控制台:

\begin{lstlisting}[style=styleCXX]
unsigned int
struct wrapper<unsigned int>
struct wrapper<struct wrapper<unsigned int> >
struct wrapper<struct wrapper<struct wrapper<unsigned int> > >
\end{lstlisting}

C++标准没有为递归嵌套的模板实例化指定限制,但推荐限制为1024个。这只是一个建议,而不是要求,所以不同的编译器实现了不同的限制。VC++ 16.11编译器的限制设置为500,GCC 12为900,Clang 13为1024。超过此限制时将使编译器报错。以下是一些例子:

VC++:

\begin{tcblisting}{commandshell={}}
fatal error C1202: recursive type or function dependency
context too complex
\end{tcblisting}

GCC:

\begin{tcblisting}{commandshell={}}
fatal error: template instantiation depth exceeds maximum of
900 (use '-ftemplate-depth=' to increase the maximum)
\end{tcblisting}

Clang:

\begin{tcblisting}{commandshell={}}
fatal error: recursive template instantiation exceeded maximum
depth of 1024
use -ftemplate-depth=N to increase recursive template
instantiation depth
\end{tcblisting}

对于GCC和Clang,可以使用编译器选项-ftemplate-depth=N可以增加嵌套模板实例化的最大值,但Visual C++编译器没有这种选项。

递归模板在编译时,可以以递归的方式解决一些问题。使用递归函数模板、变量模板,还是类模板,取决于具体问题或开发偏好。但深度模板递归的深度是有限制的,需要明智地使用模板递归。

本章要讨论的下一个高级主题是模板参数推导,包括函数和类。





