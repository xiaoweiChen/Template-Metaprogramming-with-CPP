
模板只是蓝图,编译器在遇到模板时,会根据模板创建实际代码。从模板声明中为函数、类或变量创建定义的行为称为\textbf{模板实例化}。这可以是\textbf{显式的}(告诉编译器何时应该生成定义时),也可以是\textbf{隐式的}(编译器根据需要生成新定义时)。我们将在接下来的内容来了解这两种实例化方式。

\subsubsubsection{2.5.1\hspace{0.2cm}隐式实例化}

当编译器基于模板的使用生成定义,并且没有显式实例化时,就会触发隐式实例化。隐式实例化模板定义在与模板相同的命名空间中,不过编译器从模板创建定义的方式可能有所不同。先来看看这段代码:

\begin{lstlisting}[style=styleCXX]
template <typename T>
struct foo
{
	void f() {}
};

int main()
{
	foo<int> x;
}
\end{lstlisting}

有一个名为foo的类模板,有一个成员函数f。main函数中,定义了一个类型为foo<int>的变量,但没有使用其成员。因为采用这种方式使用foo,所以编译器隐式地为int类型定义foo的特化。可以在cppinsights.io上使用在Clang运行,会看到下面的代码:

\begin{lstlisting}[style=styleCXX]
template<>
struct foo<int>
{
	inline void f();
};
\end{lstlisting}

因为函数f在代码中没有调用,所以其只是声明而没有定义。若在main中添加调用f的代码,会发生如下的特化:

\begin{lstlisting}[style=styleCXX]
template<>
struct foo<int>
{
	inline void f() { }
};
\end{lstlisting}

若在下面的实现中添加一个包含错误的函数g,会在不同的编译器中看到不同的行为:

\begin{lstlisting}[style=styleCXX]
template <typename T>
struct foo
{
	void f() {}
	void g() {int a = "42";}
};

int main()
{
	foo<int> x;
	x.f();
}
\end{lstlisting}

g的主体包含一个错误(也可以使用static\_assert(false)语句作为替代)。这段代码在VC++中编译没有问题,但是在Clang和GCC中就会失败。因为VC++忽略了模板中未使用的部分,前提是代码语法正确,但其他部分在模板实例化之前进行了语义验证。


对于函数模板,当用户代码在需要函数定义存在的上下文中引用函数时,就会发生隐式实例化。对于类模板,当用户代码在需要完整类型的上下文中引用模板时,或者当类型的完整性影响代码时,也会隐式实例化。此类上下文是构造此类类型的对象,但在声明指向类模板的指针时就是另外一种情况了。来看看下面的例子:

\begin{lstlisting}[style=styleCXX]
template <typename T>
struct foo
{
	void f() {}
	void g() {}
};

int main()
{
	foo<int>* p;
	foo<int> x;
	foo<double>* q;
}
\end{lstlisting}

使用与前面示例相同的foo类模板,并声明了几个变量:p是指向foo<int>的指针,x是指向foo<int>的指针,q是指向foo<double>的指针。由于声明了x,此时编译器只需要实例化foo<int>,考虑调用成员函数f和g:

\begin{lstlisting}[style=styleCXX]
int main()
{
	foo<int>* p;
	foo<int> x;
	foo<double>* q;
	
	x.f();
	q->g();
}
\end{lstlisting}

通过这些更改,编译器需要实例化以下内容:

\begin{itemize}
\item 
当声明x变量时,实例化foo<int>

\item 
当x.f()调用发生时,实例化foo<int>::f()

\item 
当q->g()调用发生时,实例化foo<double>和foo<double>::g()。
\end{itemize}

另外,当声明指针p时,编译器不需要实例化foo<int>;当声明指针q时,也不需要实例化foo<double>。然而,当涉及到指针转换时,编译器确实需要隐式实例化类模板特化:

\begin{lstlisting}[style=styleCXX]
template <typename T>
struct control
{};

template <typename T>
struct button : public control<T>
{};

void show(button<int>* ptr)
{
	control<int>* c = ptr;
}
\end{lstlisting}

函数show中,button<int>*和control<int>*之间进行了转换,此时编译器必须实例化button<int>。

当类模板包含静态成员时,编译器隐式实例化类模板,这些成员不会隐式实例化;只在编译器需要它们的定义时,才会隐式实例化。另外,类模板的每个特化都有自己的静态成员副本:

\begin{lstlisting}[style=styleCXX]
template <typename T>
struct foo
{
	static T data;
};

template <typename T> T foo<T>::data = 0;

int main()
{
	foo<int> a;
	foo<double> b;
	foo<double> c;
	
	std::cout << a.data << '\n'; // 0
	std::cout << b.data << '\n'; // 0
	std::cout << c.data << '\n'; // 0
	
	b.data = 42;
	std::cout << a.data << '\n'; // 0
	std::cout << b.data << '\n'; // 42
	std::cout << c.data << '\n'; // 42
}
\end{lstlisting}

类模板foo有一个名为data的静态成员变量,在foo定义之后初始化。main函数中,变量a的类型为foo<int>,b和c的类型为foo<double>。最初,它们都将成员数据初始化为0,但变量b和c共享相同的数据副本。因此,赋值b.data = 42后,a.data仍然是0,但b.data和c.data都是42。

了解了隐式实例化如何工作之后,是时候来了解显式实例化了。

\subsubsubsection{2.5.2\hspace{0.2cm}显式实例化}

可以显式地告诉编译器实例化类模板或函数模板,这称为显式实例化。它有两种形式:\textbf{显式实例化定义}和\textbf{显式实例化声明}。


\hspace*{\fill} \\ %插入空行
\noindent\textbf{定义}

显式实例化定义可以出现在程序中的任何地方,但不能出现在所引用的模板定义之后。显式模板实例化定义的语法采用以下形式:

\begin{itemize}
\item 
类模板的语法如下:
\begin{lstlisting}[style=styleCXX]
template class-key template-name <argument-list>
\end{lstlisting}

\item 
函数模板的语法如下:
\begin{lstlisting}[style=styleCXX]
template return-type name<argument-list>(parameter-list);
template return-type name(parameter-list);
\end{lstlisting}
\end{itemize}

显式实例化定义用template关键字引入,但后面没有任何参数列表。对于类模板,class-key可以是class、struct或union关键字中的一个。对于类模板和函数模板,带有给定参数列表的显式实例化定义只能在整个程序中出现一次。

我们将通过一些例子来理解这是如何工作的。下面是第一个例子:

\begin{lstlisting}[style=styleCXX]
namespace ns
{
	template <typename T>
	struct wrapper
	{
		T value;
	};
	template struct wrapper<int>; // [1]
}

template struct ns::wrapper<double>; // [2]

int main() {}
\end{lstlisting}

wrapper<T>是在ns命名空间中定义的类模板。代码中标记为[1]和[2]的语句都表示显式实例化定义,分别为wrapper<int>和wrapper<double>。显式实例化定义只能出现在与其引用的模板(如[1])相同的命名空间中,或者必须是完全限定的(如[2])。可以为函数模板编写类似的显式模板定义:

\begin{lstlisting}[style=styleCXX]
namespace ns
{
	template <typename T>
	T add(T const a, T const b)
	{
		return a + b;
	}

	template int add(int, int); // [1]
}

template double ns::add(double, double); // [2]

int main() { }
\end{lstlisting}

第二个例子与第一个例子很相似。[1]和[2]都表示add<int>()和add<double>()的显式模板定义。

若显式实例化定义与模板不在同一个命名空间中,则名称必须完全限定。using语句的使用不会使名称在当前命名空间中可见:

\begin{lstlisting}[style=styleCXX]
namespace ns
{
	template <typename T>
	struct wrapper { T value; };
}

using namespace ns;

template struct wrapper<double>; // error
\end{lstlisting}

本例中的最后一行在编译时会报错,因为wrapper是一个未知的名称,必须使用命名空间名称进行限定,如ns::wrapper。

当类成员用于返回类型或参数类型时,显式实例化定义会忽略成员访问规则:

\begin{lstlisting}[style=styleCXX]
template <typename T>
class foo
{
	struct bar {};
	
	T f(bar const arg)
	{
		return {};
	}
};

template int foo<int>::f(foo<int>::bar);
\end{lstlisting}

类X<T>::bar和函数foo<T>::f()都是foo<T>类的private部分,但它们可以在最后一行显示的显式实例化定义中使用。

了解了显式实例化定义是什么,以及它如何工作之后,现在的问题就是它什么时候有使用。为什么要告诉编译器从模板生成实例化?答案是其有助于分发库、减少构建时间和可执行文件大小。若正在构建.lib文件作为发布库,并且该库使用模板,则未实例化的模板定义不会放入库中,在每次使用库时用户代码的构建时间增加。通过强制在库中实例化模板,这些定义会放入目标文件和分发的.lib文件中。因此,用户代码只需要链接到库文件中可用的函数即可,这就是Microsoft MSVC CRT库为所有流、区域设置和字符串类所做的工作。libstdc++库会对字符串类和其他类执行相同的操作。

模板实例化可能会出现的问题是,最终可能会得到多个定义,每个翻译单元一个定义。若包含模板的同一个头文件包含在多个翻译单元(.cpp文件)中,并且使用了相同的模板实例化(假设wrapper<int>来自前面的例子),这些实例化的副本放在每个翻译单元中,这将导致目标文件大小的增加。这个问题可以在显式实例化声明的帮助下解决。

\hspace*{\fill} \\ %插入空行
\noindent\textbf{声明}

显式实例化声明(C++11中提供)可以告诉编译器模板实例化的定义在不同的翻译单元中,并且不应该生成新的定义。语法与显式实例化定义相同,除了关键字extern可用在声明前:

\begin{itemize}
\item 
类模板的语法如下所示:
\begin{lstlisting}[style=styleCXX]
extern template class-key template-name <argument-list>
\end{lstlisting}

\item 
函数模板的语法如下所示:
\begin{lstlisting}[style=styleCXX]
extern template return-type name<argumentlist>(parameter-list);
extern template return-type name(parameter-list);
\end{lstlisting}
\end{itemize}

若提供了显式的实例化声明,但在程序的翻译单元中都不存在实例化定义,则结果是编译器警告和链接器错误。该技术是在一个源文件中声明显式模板实例化,在其余的源文件中声明显式模板声明,这将减少编译时间和目标文件的大小。

来看看下面的例子:

\begin{lstlisting}[style=styleCXX]
// wrapper.h
template <typename T>
struct wrapper
{
	T data;
};

extern template wrapper<int>; // [1]

// source1.cpp
#include "wrapper.h"
#include <iostream>

template wrapper<int>; // [2]

void f()
{
	ext::wrapper<int> a{ 42 };
	std::cout << a.data << '\n';
}

// source2.cpp
#include "wrapper.h"
#include <iostream>

void g()
{
	wrapper<int> a{ 100 };
	std::cout << a.data << '\n';
}

// main.cpp
#include "wrapper.h"

int main()
{
	wrapper<int> a{ 0 };
}
\end{lstlisting}

这个例子中,可以看到:

\begin{itemize}
\item 
wrapper.h头文件包含一个名为wrapper<T>的类模板。[1]中,wrapper<int>有一个显式的实例化声明,告诉编译器在编译包含此头文件的源文件(翻译单元)时,不要为该实例化生成定义。

\item 
source1.cpp文件包含wrapper.h,并且[2]中包含wrapper<int>的显式实例化定义。这是整个程序中这个实例化的唯一定义。

\item 
源文件source2.cpp和main.cpp都使用了wrapper<int>,但是没有显式的实例化定义或声明。当头文件包含在每个文件中时,wrapper.h的显式声明是可见的。
\end{itemize}

或者,可以从头文件中取出显式实例化声明,但随后必须将其添加到包含头文件的每个源文件中,而这很可能被遗忘。

进行显式模板声明时,类主体中定义的类成员函数总是视为inline,因此会实例化,所以只能对在类体之外定义的成员函数使用extern关键字。

现在已经了解了什么是模板实例化,我们将继续讨论另一个重要的主题:模板特化。这个术语用于从模板实例化创建的定义,以处理一组特定的模板参数。















