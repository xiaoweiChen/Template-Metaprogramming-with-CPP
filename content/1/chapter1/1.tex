每种语言特性都旨在帮助开发人员解决当前的问题或任务,模板的目的是为了避免编写只有细微差别的重复性代码。

这里,我们以一个经典的max函数为例。函数接受两个数值参数,并返回两个参数中最大的一个:

\begin{lstlisting}[style=styleCXX]
int max(int const a, int const b)
{
	return a > b ? a : b;
}
\end{lstlisting}

这个函数没毛病,但只适用于int类型的值(或那些可转换为int的值)。若需要相同的函数,但参数类型是double怎么办呢?直接为double类型重载这个函数就好了(创建一个具有相同名称,但参数数量或类型不同的函数):

\begin{lstlisting}[style=styleCXX]
double max(double const a, double const b)
{
	return a > b ? a : b;
}
\end{lstlisting}

然而,int和double并不是唯一的数字类型。有char, short, long, long和相应的unsigned版本,unsigned char, unsigned short, unsigned long, unsigned long。还有浮点型和长双精度型,以及其他类型,如int8\_t, int16\_t, int32\_t和int64\_t。可以比较其他类型,例如bigint、Matrix、point2d和任何支持大于运算符的自定义类型。通用库如何为这些类型提供诸如max这样的通用函数呢?它可以重载所有内置类型和其他库类型的函数,但不能重载自定义类型。

使用不同参数重载函数的另一种方法是使用void*来传递不同类型的参数,但这很糟糕。下面的示例只是在没有模板的情况下作为一种可能的替代方案。这里,我们设计了一个排序函数,该函数将对任何可能类型的元素数组运行快速排序,该数组提供严格的弱排序。快速排序算法的详细信息可以在网上查询,例如:\url{https://en.wikipedia.org/wiki/Quicksort}。

快速排序算法需要比较和交换两个元素,但由于事先不知道其数据类型,不能直接实现。解决方案是依赖回调,回调是作为参数传递的函数,可以在必要时调用:

\begin{lstlisting}[style=styleCXX]
using swap_fn = void(*)(void*, int const, int const);
using compare_fn = bool(*)(void*, int const, int const);

int partition(void* arr, int const low, int const high,
			  compare_fn fcomp, swap_fn fswap)
{
	int i = low - 1;
	
	for (int j = low; j <= high - 1; j++)
	{
		if (fcomp(arr, j, high))
		{
			i++;
			fswap(arr, i, j);
		}
	}

	fswap(arr, i + 1, high);
	
	return i + 1;
}

void quicksort(void* arr, int const low, int const high,
			   compare_fn fcomp, swap_fn fswap)
{
	if (low < high)
	{
		int const pi = partition(arr, low, high, fcomp,
			fswap);
		quicksort(arr, low, pi - 1, fcomp, fswap);
		quicksort(arr, pi + 1, high, fcomp, fswap);
	}
}
\end{lstlisting}

为了调用快速排序函数,需要为传递给函数的每种类型的数组提供比较和交换函数的实现。以下是int类型的实现:

\begin{lstlisting}[style=styleCXX]
void swap_int(void* arr, int const i, int const j)
{
	int* iarr = (int*)arr;
	int t = iarr[i];
	iarr[i] = iarr[j];
	iarr[j] = t;
}

bool less_int(void* arr, int const i, int const j)
{
	int* iarr = (int*)arr;
	return iarr[i] <= iarr[j];
}
\end{lstlisting}

定义了所有这些后,可以编写如下代码对整数数组进行排序:

\begin{lstlisting}[style=styleCXX]
int main()
{
	int arr[] = { 13, 1, 8, 3, 5, 2, 1 };
	int n = sizeof(arr) / sizeof(arr[0]);
	quicksort(arr, 0, n - 1, less_int, swap_int);
}
\end{lstlisting}

这些例子关注的是函数,同样的问题也适用于类。假设要编写一个类,可对具有可变大小的数值集合建模,并将元素连续存储在内存中。可以使用以下实现(这里仅概述声明)来存储整数:

\begin{lstlisting}[style=styleCXX]
struct int_vector
{
	int_vector();
	
	size_t size() const;
	size_t capacity() const;
	bool empty() const;
	
	void clear();
	void resize(size_t const size);
	
	void push_back(int value);
	void pop_back();
	
	int at(size_t const index) const;
	int operator[](size_t const index) const;
private:
	int* data_;
	size_t size_;
	size_t capacity_;
};
\end{lstlisting}

目前这一切看起来都很好,但当需要存储类型为double、std::string或自定义类型的值时,将需要编写相同的代码,每次只更改元素的类型。应该没有人愿意干这种事情吧,这明显就是一项重复性极高的工作,而且当需要更改某些内容时(例如添加新功能或修复错误),就需要在多个地方进行相同的更改。

最后,当要定义变量时,可能会遇到类似的问题(尽管不太常见)。考虑保存换行字符的变量的情况:

\begin{lstlisting}[style=styleCXX]
constexpr char NewLine = '\n';
\end{lstlisting}

若需要相同的常量,但需要不同的编码,比如:宽字符串字面值、UTF-8等,该怎么办?多个变量当然可以,有不同的名字,比如下面的例子:

\begin{lstlisting}[style=styleCXX]
constexpr wchar_t NewLineW = L'\n';
constexpr char8_t NewLineU8 = u8'\n';
constexpr char16_t NewLineU16 = u'\n';
constexpr char32_t NewLineU32 = U'\n';
\end{lstlisting}

模板是一种技术手段,其允许开发人员编写蓝图,使编译器能够为我们生成所有这些重复的代码。下一节中,我们将看到如何将前面的代码转换为C++模板。






